\makeatletter
\def\thickhrulefill{\leavevmode \leaders \hrule height 1ex \hfill \kern \z@}
\def\@makechapterhead#1{%
  \vspace*{10\p@}%
  {\parindent \z@ \centering \reset@font
        {\Huge \scshape \thechapter}
        \par\nobreak
        \vspace*{15\p@}%
        \interlinepenalty\@M
        \begin{tabular}{@{\qquad}c@{\qquad}}
          \hline
          \\
          {\Huge \bfseries #1\par\nobreak} \\
          \\
          \hline
        \end{tabular}
    \vskip 100\p@
  }}
\def\@makeschapterhead#1{%
  \vspace*{10\p@}%
  {\parindent \z@ \centering \reset@font
        {\Huge \scshape \vphantom{\thechapter}}
        \par\nobreak
        \vspace*{15\p@}%
        \interlinepenalty\@M
        \begin{tabular}{@{\qquad}c@{\qquad}}
          \hline
          \\
          {\Huge \bfseries #1\par\nobreak} \\
          \\
          \hline
        \end{tabular}
    \vskip 100\p@
  }}

\chapter{Szak térképek ábrázolási módszerei}
  A következőkben definiálni fogjuk a topográfiai térkép fogalmát:
	
	\section{Szaktérképek}
	A következőkben a  szaktérképekről szeretnék egy keveset ismertetni.
		
	A térképi elemek jelölik azokat a tárgyakat és jelenségeket, amelyek a térképre kerülnek, az ábrázolás maga,
	ábrázolási elemek segítségével történik. Az ábrázolási elemek együttese adja a térképtartalmat. A szaktérkép
	önmagában egy zárt, a valóságot a mennyiségi, minőségi, előfordulás, intenzitás, időpont vagy időtartam szerint
	ábrázolhatjuk.
		
	A szaktérképek elvi kivitelezésüket tekintve négy alapvető csoportba sorolhatóak:
	\begin{enumerate}
		\item Analitikus szaktérképek
		\item Komplex szaktérképek
		\item Szintetikus szaktérképek
		\item Szinoptikus szaktérképek
	\end{enumerate}
		
	\subsection{Analitikus szaktérképek}
	Jelenségeket, tényeket, folyamatokat stb. egy fölérendelt formai vagy funkcionális szerkezet részeként bontjuk fel
	elemekre.
	
	\subsection{Komplex szaktérképek}
	Egymás mellett többfajta jelenséget, tényt, folyamatot stb. ábrázolunk, azaz több analitikus térképet szeretnénk
	egy térképen szemléltetni.
	
	\subsection{Szintetikus szaktérképek}
	Ezeken a térképeken a formai vagy funkcionális szerkezeten belül a valóságot szintetikus összetevő, egységesítő
	jelleggel ábrázoljuk. Analitikus $+$ Komplex  $=$ Szintetikus. Pl.: tájtípustérkép.
	
	\subsection{Szinoptikus szaktérképek}
	Többfajta jelenséget áttekintő, összefoglaló módon ábrázol, s így együttes hatásukat is szemlélteti. 
	Pl.: tájtípusok, klímatípusok egymásra hatása, régiók.
		
	\section{Térképtípusok}
	A térképek tervezésénél és megjelenítésénél sok szempontot kell figyelembe venni, mint az átláthatóság, egyértelműség
	és közlékenység. Egy nagyon fontos tényező, amely még mindig fejfájást okoz egy szaktérkép szerkesztésénél, az 
	\emph{idő}, azaz a negyedik dimenzió. Az ábrázolás megoldásával kapcsolatos próbálkozások többirányúak voltak, és 
	néhány új térképtípust eredményeztek:
	\begin{enumerate}
		\item Statikus térképek
		\item Dinamikus térképek
		\item Genetikus térképek
	\end{enumerate}
	
	\subsection{Statikus térképek}
	Egy bizonyos folyamatot, jelenséget, egy megadott időpontra jellemző állapotában ábrázolják.
	
	\subsection{Dinamikus térképek}
	Egy bizonyos folyamatot, jelenséget, egy megadott időpontra jellemző állapotában ábrázolják, de
	jelekkel, nyilakkal, számokkal a dinamikáját próbálják érzékeltetni.
	
	\subsection{Genetikus térképek}
	A fejlődés térképek vagy változástérképek bonyolult jelenségeket, folyamatokat stb. igyekeznek keletkezésükben
	fejlődésükben négydimenziós formájában ábrázolni.
	
	\section{Áttekintés}
	A négy szaktérképcsoportok (analitikus, komplex, szintetikus, szinoptikus), továbbá statikus, dinamikus és genetikus
	jelleg, és a mennyiségi, minőségi ábrázolás variációiból, és az ábárázolandó megjelenítési lehetőségeiből adódik, hogy
	a szaktérképek szerkesztési skálája elméletileg végtelen, azaz a legkülönbözőbb témákból végtelen sok térképet lehet szerkeszteni.
	
	A generalizálási szabályok betartásával és a tartalom mennyiségi illetve minőségi arányok figyelembe vételével lehet
	meghatározni az ábrázolás lehetőségét és jellegét. A térképi ábárzolás elemei:
	\begin{itemize}
		\item pont
		\item vonal
		\item terület
		\item térhatású jelek
		\item színvariációk
		\item szöveges leírások
	\end{itemize}
	Gondos és átgondolt ábrázolási elemek együttese (jó tartalmi és technikai generalizálás) biztosítja az objektív valóságot,
	jól olvashatóságot, és nem utolsó sorban esztétikailag kellemes megjelenésű térképet.
	
	A jelkulcs megszerkesztésekor, a jeleket úgy kell megválasztni, hogy a felhasználó különösebb gondolkodás nélkül
	ráismerjen a témára, amelyet a jel szimbolizál. Ha ez valamilyen oknál fogva nem oldható meg, akkor vissza kell térni az alapvető
	könnyen érthető jelekhez, mint a kör, háromszög.
	
	Fontos, hogy a jelek az első pillantásra megkülönböztethetőek legyenek, azonkívül egységesnek kell lenniük, azaz
	azonos tulajdonságokat (minőségi) lehetőleg azonos nagyságú és formájú jelekkel ábárzoljuk.
