\makeatletter
\def\thickhrulefill{\leavevmode \leaders \hrule height 1ex \hfill \kern \z@}
\def\@makechapterhead#1{%
  \vspace*{10\p@}%
  {\parindent \z@ \centering \reset@font
        {\Huge \scshape \thechapter}
        \par\nobreak
        \vspace*{15\p@}%
        \interlinepenalty\@M
        \begin{tabular}{@{\qquad}c@{\qquad}}
          \hline
          \\
          {\Huge \bfseries #1\par\nobreak} \\
          \\
          \hline
        \end{tabular}
    \vskip 100\p@
  }}
\def\@makeschapterhead#1{%
  \vspace*{10\p@}%
  {\parindent \z@ \centering \reset@font
        {\Huge \scshape \vphantom{\thechapter}}
        \par\nobreak
        \vspace*{15\p@}%
        \interlinepenalty\@M
        \begin{tabular}{@{\qquad}c@{\qquad}}
          \hline
          \\
          {\Huge \bfseries #1\par\nobreak} \\
          \\
          \hline
        \end{tabular}
    \vskip 100\p@
  }}

\chapter{Térképszerkesztés, térképtervezés}
\label{sec:térképészet}
% Define the first section 

	A következőkben a térképszerkesztés illetve-tervezéssel kapcsolatos feladatokról, követelményekről lesz szó. Az egyik
	legfontosabb témakör a generalizálás, amely minden térképtípusnál különös figyelmet igényel. Ezt sajnos sokan
	nem veszik figyelembe. A következő fejezetben szeretnék néhány dolgot
	kiemelni, illetve részletesebben tárgyalni.
		%Brief Introduction on section
		\section{Generalizálás}
		Valamilyen cél érdekében egy adott méretaránynak megfelelően generalizálunk, egyszerűsítjük a térképi tartalmat,
		egyúttal előkészítjük a kialakítandó
		térkép ábrázolási rendszerét is. Grafikus megjelenítéskor, gyakorlatilag a jelkulcsokat definiáltuk az ábrázolási rendszerrel.
		
		\subsection{Alapfogalmak}
		Térkép típusoktól függően más és más generalizálási igények merülnek fel, ezeket a későbbiekben fogjuk tárgyalni.

		\begin{definition}
			Térkép, lehet információrögzítő, közvetítő vagy tájékoztató. Más esetben pedig információforrásként szolgál,
			amely az objektív valóság megismeréséhez nyújthat segítséget.
		\end{definition}

		\begin{definition}
			Az egyensúly megtartása. Az információ mennyiség és egyértelmű áttekintés, megértés, valamint minőség között
			mindig megkell találni az egyensúlyt.
		\end{definition}
		
  	    \begin{definition}[Topográfiai térkép]
		  	Azon térképeket nevezzük topográfiai térképnek, amelynek a síkrajz, vízrajz, felszíni formák,
		növénytakaró és egy sor egyéb, az általános tájékozódáshoz szükséges feltüntetett tárgy a fő eleme és
		amelyet a névrajz részletesen magyaráz.
		  \end{definition}
		  Ide tartoznak a 
			\begin{enumerate}
				\item Generalizált nagy-, közép- vagy kisméretarányú topográfiai térkép.
				\item Célgeneralizált nagy-, közép- vagy kisméretarányú topográfiai térkép.
				\item Szaktérképek, ezzel fogunk most részletesebben foglalkozni.
		\end{enumerate}

		\begin{definition}[Szaktérképek]
			Egy térképi alapon az objektumok, tények, jelenségek, folyamatok síkban, térben elhelyezkedő együttesét
			, azaz természeti és társadalmi jelenségek szerkezetét, funkcióját, egymásrahatását, adatok, analízisekből és
			szintézisekből kapott értékek révén, múltra, jelenre, jövőre vonatkozóan vagy együttesen mutatják be, mint információhordozók
			- sajátosan generalizált módon - s ezáltal újabb információk, inspirációk forrásaivá válhatnak.
		\end{definition}

		
		\subsection{Szempontok}
		Generalizálás folyamata során az ábrázolási mód szab határt az egyszerűsítés mértékének. Például két hasonló méretarányban lévő térképnél
		de jellegében más típusúak, eltérhetnek adat befogadó képesség szempontjából. A következő generalizálási módokról
		beszélhetünk az adatok oldaláról vizsgálva:
		\begin{itemize}
			\item mennyiségi
			\item minőségi
			\item térbeli
			\item időbeli
		\end{itemize}
		Ezeket nem tudjuk teljes mértékben egyenrangúan érvényesíteni. Ezért válogatni kell az ábrázolási lehetőségek szerint,
		ezt a folyamatot nevezzük generalizálásnak.
		
		
		A generalizálás más szóval lehet kiválasztás, válogatás, egyszerűsítés vagy általánosítás.
		
		\subsection{Idézetek}
		Két idézetet szeretnék megemlíteni, amelyek megvilágítják a kérdéskört.
		\begin{enumerate}
			\item Imhof: "Minden kartográfiai generalizálás arra törekszik, hogy a már lekicsinyített és olvashatatlanná
			vált képet olvashatóvá tegye. Ez bizonyos sajátosságok összevonásával, kihagyásával és kiemelésével
			érhető el."
			\item Robinson: "Minden térképen, amely a topográfiai méretarányoknál kisebb, ki kell választani az ábrázolandó
			tárgyakat, egyszerűsíteni kell formájukat és értékelni kell a célnak megfelelő viszonylagos jelentőségüket, hogy 
			a fontos elemek jobban kiemelkedjenek. Ez a folyamat a kartográfiai generalizálás."
		\end{enumerate}
		
		\subsection{Kifejezőképesség}
		Egy térkép kifejező erejét a tartalmi generalizálás, valamint a rajzi megoldások határozzák meg.  Innen látjuk, hogy 
		a generalizálás folyamatnak két aspektusa van:
		\begin{enumerate}
			\item tartalmi: térkép tartalmi, méretarány, objektumok szelektálása, összevonása, kiemelése, osztályozását, amely
			az ábrázolási jellegét adja.
			\item formai: térkép lokalizálását, határvonalak, kontúrok, névrajzi anyagok összhangjára öszpontosít a formai nézet.
		\end{enumerate}
		
		%Nyolc alapszabály generalizálásra
		\subsection{Hét alapszabály}
		A generalizálást hét alapszábállyal lehet jellemezni:
		\begin{itemize}
			\item Mértani
			\begin{enumerate}
				\item az egyszerűsítés
				\item a nagyobbítás
				\item az eltolás
			\end{enumerate}
			\item Mennyiségi
			\begin{enumerate}
				\item az összevonás
				\item a kiválasztás
			\end{enumerate}
			\item Minőségi
			\begin{enumerate}
				\item a tipizálás
				\item a hangsúlyozás
			\end{enumerate}
		\end{itemize}
		
		\section{Töpfer-féle gyökszabály}
		Úgy gondolom, elengedhetetlen megemlíteni a Töpfer-féle gyökszabályt, amely a mennyiségi oldalról
		közelíti meg a generalizálás kérdését. Az alábbi formula adja a szabály alapját:
		\[
		n_F = n_A \sqrt{\frac{M_A}{M_F}}
		\], ahol
		\begin{center}
			\begin{tabular}{ll}
			$n_F$=&objektumok mennyisége a készítendő kérképen\\
			$n_A$=&objektumok száma az alaptérképen\\
			$M_F$=&levezetett új méretarány\\
			$M_A$=&alaptérkép méretaránya\\
			\end{tabular}			
		\end{center}
		Azonban, a szabályt csak topográfiai térképeknél alkalmazható, kisebb méretarányú térképek esetén nem alkalmazható.

		\section{Jelkulcs és jelmagyarázat}
		
		A jelkulcsnak kifejezőnek kell lennie, hogy kiemelje az ábrázolandó térkép lényegét, tematikáját. Ahhoz, hogy jól meghatározzuk
		a jelkulcsot, két szempontot kell megvizsgálnunk. Az egyik a tárgyi tövényszerűségek, amelyek az ábrázolandó objektumok ismérvei,
		a másik a rajzi törvényszerűség, amelyek az ábrázolási módot határozza meg.
		
		\subsection{Tárgyi törvényszerűségek}
		Az alábbiakban szempontokat sorolok fel:
		\begin{itemize}
			\item Minőség és mennyiség: hol mi van? illetve hol mennyi van?
			\item Diszkrétumok és kontinumok: térben lehatárolhatók (pl. tó, erdő), illetve térben nem határolhatók (pl. csapadék, hőmérséklet).
			\item Statikus és dinamikus: egy adott állapotot írnak le, vagy a változást szemléltetik.
			\item Eredeti és levezetett objektumok: eredeti objektumok, észlelhető vagy egy terület térbeli és időbeli változása.
			\item Jelenségek és tényállások: objektumok, illetve gondolati fogalmak.
		\end{itemize}
		\subsection{Rajzi törvényszerűség}
		Az alábbiakban szempontokat sorolok fel:
		\begin{itemize}
			\item Térképelemek, ábrázolási elemek minimális méretei.
			\item Az ábrázolás helyzethűsége, alaprajzhoz hasonlósága.
		\end{itemize}
		Fontos a közepes és kisméretarányú térképeknél a tartalom generalizálása. Tartalommal folyton bővülő térképek esetében megfelelő 
		egyedi jelkulcsot készítünk, az azonos elemeket egységes módon kell a jelkulcsba szerkeszteni, ezzel az egységességet és összetartozást
		reprezentáljuk.
		
		A jelkulcs a térkép kszítője számára készül.
		
		A jelkulcsok jelmagyarázatát rövid, tömörségre kell törekedni. A terjengős jelmagyarázatok zavarossá, áttekinthetetlenné teszik az égész
		tartalmi anyagot. A jelmagyarázati szöveget általában egyesszámban írjuk és nagybetűvel kezdjük. A jelmagyarázatban témákra bontjuk a
		térképi tartalmat, a sorrendiségre nincs kiírás. A jelmagyarázatok lehetőleg legyenek fix, vagy kivehetőek a térképből.
