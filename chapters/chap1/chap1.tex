\makeatletter
\def\thickhrulefill{\leavevmode \leaders \hrule height 1ex \hfill \kern \z@}
\def\@makechapterhead#1{%
  \vspace*{10\p@}%
  {\parindent \z@ \centering \reset@font
        {\Huge \scshape \thechapter}
        \par\nobreak
        \vspace*{15\p@}%
        \interlinepenalty\@M
        \begin{tabular}{@{\qquad}c@{\qquad}}
          \hline
          \\
          {\Huge \bfseries #1\par\nobreak} \\
          \\
          \hline
        \end{tabular}
    \vskip 100\p@
  }}
\def\@makeschapterhead#1{%
  \vspace*{10\p@}%
  {\parindent \z@ \centering \reset@font
        {\Huge \scshape \vphantom{\thechapter}}
        \par\nobreak
        \vspace*{15\p@}%
        \interlinepenalty\@M
        \begin{tabular}{@{\qquad}c@{\qquad}}
          \hline
          \\
          {\Huge \bfseries #1\par\nobreak} \\
          \\
          \hline
        \end{tabular}
    \vskip 100\p@
  }}

\chapter{Bevezetés}

  A Térképet mint megjelenítési eszközt szeretném felhasználni mivel rendkívül nagy a kifejező ereje. Ennek segítségére szolgál az 
informatika. E két terület együttes használatával, egy térképes alapú információs rendszert szeretnék bemutatni.

  Térképészet vagy kartográfia (a görög khartisz = hártya, kártya és \\ graphein = írás, rajzolás)  a térképek és más kartográfiai 
ábrázolási formák (például földgömb, panorámakép, fototérkép stb.) készítésének és használatának tudománya, technikája és művészete.
