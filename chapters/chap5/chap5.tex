\makeatletter
\def\thickhrulefill{\leavevmode \leaders \hrule height 1ex \hfill \kern \z@}
\def\@makechapterhead#1{%
  \vspace*{10\p@}%
  {\parindent \z@ \centering \reset@font
        {\Huge \scshape \thechapter}
        \par\nobreak
        \vspace*{15\p@}%
        \interlinepenalty\@M
        \begin{tabular}{@{\qquad}c@{\qquad}}
          \hline
          \\
          {\Huge \bfseries #1\par\nobreak} \\
          \\
          \hline
        \end{tabular}
    \vskip 100\p@
  }}
\def\@makeschapterhead#1{%
  \vspace*{10\p@}%
  {\parindent \z@ \centering \reset@font
        {\Huge \scshape \vphantom{\thechapter}}
        \par\nobreak
        \vspace*{15\p@}%
        \interlinepenalty\@M
        \begin{tabular}{@{\qquad}c@{\qquad}}
          \hline
          \\
          {\Huge \bfseries #1\par\nobreak} \\
          \\
          \hline
        \end{tabular}
    \vskip 100\p@
  }}

\chapter{Fejleszői Dokumentáció}
	\section{Technológia Áttekintés}
	Geo-wiki megírásához különböző komponensek együttműködését igényli. Az adatbázis motor, a web-es keretrendszer, Google maps API, verzió követő rendszer kiválasztása mind
	a fejlesztési fázis elő szakaszába tartozik. A továbbiakban szeretném bemutatni az előkészítési fázis nehézségeit és az okát, hogy miért döntöttem
	a kiválasztott technológiák mellett.
	
	Első gondolatom, az volt, hogyan tudnám a legkevesebb kódból megírni a Geo-Wiki. Első próbálkozásom az open source alapú wiki motorokra öszpontosult, hogy
	megspóroljam a tartalom bevitel és megjelenítő réteg megírásához szükséges időt.
	Ezek között lényeges különbségek a rendszer rugalmasságában észlelhető. Az alábbiakban felsorolnék egy párat közülük:
	\begin{center}
	    \begin{tabular}{ | l | l | l | l | l | p{5cm} |}
		\hline
	    \hline
	    Név & Alkotó & Nyelv & Adatkezelő  & Nyílt   \\ \hline
	    Media Wiki & Wikipedia Foundation & PHP & RDBMS & Open\\ \hline
	    TWiki & Peter Thoeny & Perl & File & Enterprise \\ \hline
	    Doku Wiki & Andreas Gohr & PHP &  File & Open\\ \hline
		Instiki &  David Heinemeier Hansson & RoR & RDBMS & Open \\ \hline
	    \hline
	    \end{tabular}
	\end{center}
	Mint láthatjuk, a motorok többsége PHP alapú. Kipróbálásuk után éreztem, hogy további komponensekkel kibővítve valószínűleg
	rengeteg erőfeszítést és időt igényel mivel egy ismeretlen keretrendszert kell megismerni. Ezért úgy döntöttem, hogy elkezdem nulláról írni az alkalmazást.
	
	A következő lépés az adatbázis motor kiválasztása, mivel az információkat elkell tárolni. Itt két szempontot vettem figyelembe:
	\begin{enumerate}
		\item GIS funkcionalitás.
		\item Könnyű, gyors és nyílt forrás kódú.
	\end{enumerate}
	
	Végül, megkellett néznem milyen segéd library-k, illetve wrapper-ek állnak rendelkezésre a Google Maps használatában.
	A google maps jelenleg V2 és V3 verziókban futnak. Nem véletlenül történt a főverzió váltás. A V3 sokkal többet nyújt, kibővült az API
	library. Úgy döntöttem, hogy a V3-t fogom használni annak ellenére, hogy a segéd anyagok szinte csak V2-re állnak rendelkezésre.

	% section technologiai_attekintes (end)
	\section{Keretrendszerek}
	A keretrendszer kiválasztásakor a következő szempontok kerültek előtérbe:
	\begin{itemize}
		\item Web alapú: Webes keretrendszer.
		\item Moduláris: Egy egyszerű keretrendszer, amelyhez később igény szerint más komponensek csatlakoztathatóak.
		\item Rugalmas: Bővíthetőség valamit külső nem keretrendszerbeli komponensekkel való együttműködés.
		\item Google Maps kompatibilis: I/O adatok felületet bíztosítson.
		\item Grafikus elemek kezelése.
	\end{itemize}
	Ezen szempontokat figyelembe véve a következő keretrendszereket vizsgáltam meg:
	\begin{enumerate}
		\item Akelos
		\item JEE
		\item Flash CS4
		\item Ruby on Rails
	\end{enumerate}
	
		\subsection{Akelos} % (fold)
		\label{sub:akelos}
		% content section
		
		\begin{wrapfigure}{r}{0.4\textwidth}
		  \vspace{-20pt}
		  \begin{center}
		    \includegraphics[width=0.20\textwidth]{chapters/chap5/img/akelos.jpg}
		  \end{center}
		  \vspace{-20pt}
		  \caption{Akelos}
		\end{wrapfigure}
		
		MVC (Model, Controller, View) keretrendszer PHP-ban. 
		\\
		Magas szintű API-k. Ruby On Rails alapján lett klónozva, feleslegesnek éreztem, hogy
		egy másik szkript nyelvet tanuljak meg, ahhoz, hogy ugyanazt elérjem Ruby On Rails-ben, amit már ismerek, így nem esett rá a választás. 
		% subsection symphony (end)

		\subsection{JEE} % (fold)
		Java Enterprise termékek vizsgálati céljából JBoss rendszer válaszottam, mivel kiforrott, és megbízható.\\
		\label{sub:jee}
		
		\begin{wrapfigure}{r}{0.4\textwidth}
		  \vspace{-20pt}
		  \begin{center}
		    \includegraphics[width=0.30\textwidth]{chapters/chap5/img/jboss.jpg}
		  \end{center}
		  \vspace{-20pt}
		  \caption{JBoss Redhat}
		\end{wrapfigure}
		

		Nagyon igéretesnek tűnt, főleg Hibernate rugalmassága és Java nyelv miatt. A következő komponenseit néztem meg:
		\begin{enumerate}
			\item Hibernate: rugalmas adat generálási és elérési réteg.
			\item Open Faces: felhasználói komponensek könyvtára.
			\item JBoss Application Server: alkalmazás szerver.
			\item Google Maps plugin: kiegészítő modul google maps-hez.
		\end{enumerate}
		Azonban a Faces engine kiszámíthatatlan viselkedése javascript-ekkel, le kellett mondanom. Mivel Google Maps API-ja javascriptben íródott
		, használata javascript megbízható működését igényel.
		% subsection  (end)
		
		\subsection{Flash CS4} % (fold)
		\label{sub:flash_cs4}
		
		\begin{wrapfigure}{r}{0.4\textwidth}
		  \vspace{-20pt}
		  \begin{center}
		    \includegraphics[width=0.2\textwidth]{chapters/chap5/img/fl.png}
		  \end{center}
		  \vspace{-20pt}
		  \caption{Adobe Flash}
		\end{wrapfigure}

		Grafikai szempontokat teljesen jól teljesítette, azonban egy nagyon nagy problémája van. Az adatok elérése adatbázison
		keresztül nagyon körülményes, számomra használhatatlan. Adatbázis kezelés php file-okon keresztül lehet megvalósítani POST/GET
		metódusokon keresztül.
		% subsection flash_cs4 (end)
		
		\subsection{Ruby On Rails} % (fold)
		\label{sub:ruby_on_rails}
		
		\begin{wrapfigure}{r}{0.4\textwidth}
		  \vspace{-20pt}
		  \begin{center}
		    \includegraphics[width=0.2\textwidth]{chapters/chap5/img/rails.jpg}
		  \end{center}
		  \vspace{-20pt}
		  \caption{Ruby On Rails}
		\end{wrapfigure}

		Ruby On Rails egy MVC alapú keretrendszer. 
		
		
		\emph{Model}: Active Record osztály a model réteg magja. Ezen keresztül szinte bármilyen adatbázissal tudunk dolgozni. 
		Alkalmazás írása közben is meg van a lehetőség, hogy adatbázis motort változtassunk mivel a kapcsolatot egy yml fileban van tárolva
		és Migration modul segítségével bármikor vissza tudjuk állítani a sémát egy másik adatbázison.
		
		
		\emph{View}: Active View osztály felel a megjelenítésért. html.erb kiterjesztés megadja a lehetőséget, hogy egyszerre natív html, javascript
		valamint ruby kódot tudjunk futtatni. 
		
		
		\emph{Controller} Moduláris komponensekre lehet osztani az alkalmazást. Minden egyes kontroller saját funkcionalitásáért felelős, igy struktúráltabb
		és átláthatóbb a kód.

		% subsection ruby_on_rails (end)
	% section keretrendszerek (end)
	\section{Adatbázis motor}
	Hárman indultak a versenyben, nevezetesen: Spatialite, MySQL valamint PostgreSQL.
	
	Oracle-t szándékosan nem vizsgáltam meg, mivel nem nyílt forrás kódú, azonban van egy ingyenes Express változata, amelyre 4GB file korlátozást szabtak meg.
	
		\subsection{MySQL} % (fold)
		\label{sub:mysql}
		Egy nagyon fontos kritérium, a helyes működéshez MyISAM szerkezetűnek kell lennie minden táblának. MyISAM támogatást ad egyaránt spatial és nem spatial adatokra.
		
		Spatial Extension funkciók:
		\begin{itemize}
			\item Geometria model támogatás: polygon, multipoint, linestring, curve, \ldots
			\item Adatok formátumok WKT (Well known text), WKB (Well known binary)
			\item Geometria függvények.
		\end{itemize}
		Úgy éreztem használati szempontjából egy kicsit körülményes volt.
		% subsection mysql (end)

		\subsection{Spatialite} % (fold)
		\label{sub:spatialite}
		Gyors, egyszerű, kicsi, hordozható. Egy hátránya van, nem képes konkurensen működni. Úgy gondolom fontos megemlíteni, mert
		nagyon hatékonyan lehet adatokat tárolni, rengeteg nyelven írtak API-t SQLite adatbázishoz. Spatialite extension-je pedig szinte minden
		geometria függvényt lefed, amit MySQL adatbázis motor szolgáltat.
		% subsection spatialite (end)
		
		\subsection{PostgreSQL - Postgis} % (fold)
		\label{sub:postgresql}
		Postgres adatázishoz külön telepíthető komponens. Hatékony, nyíltforráskódú adatbázisrendszer.
		
		Spatial extension funkciók:
		\begin{itemize}
			\item Geometria model támogatás: teljes.
			\item Adat formátumok: WKT, WBT.
			\item Geometria függvények.
			\item Adatkezelő felület: ERSI, SVG adatok import és export funkciók.
			\item Bulk Loader: tömeges adatok feltöltése.
			\item Third party szoftver támogatás: QGis, Grass, \ldots
		\end{itemize}
		Három adatbázis motor közül PostreSQL-re esett a választásom.
		% subsection postgresql (end)
		
	% section adatbázis_motors (end)
	\section{Google Maps}
	Lars Rasmussen és csapata által kifejlesztett Google Maps-t szeretném felhasználni. V3 a következő funkciókkal bővült V2-hez képest:
	\begin{itemize}
		\item KML adatok kezelése.
		\item Saját térkép definiálása.
		\item Geokódolásban hátékonyabb lett.
		\item Útvonal keresési funkciók hatékonysága.
		\item Google Maps Marker Manager.
	\end{itemize}
	Javascript API-n keresztül lehet elérni a szolgáltatásokat.
	
	\section{Verzió követés}
	Verzió követő rendszernek GitHub-t használom.
	
	Előnyei:
	\begin{itemize}
		\item Ingyenes.
		\item Web-es.
		\item Git verzió követő rendszert használ.
		\item Kényelmes felhasználói felület.
		\item Biztonságos.
	\end{itemize}
	
	Két repository-t hoztam létre:
	\begin{enumerate}
		\item ntb-Nagyprogram
		\item 
	\end{enumerate}
